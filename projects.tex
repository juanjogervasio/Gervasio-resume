%\begin{tabular}{p{11em} p{8em} p{13em} p{8em} p{7em}}
%\skills{Herramientas IT} & Python & \pointskill{}{4} & Excel & \pointskill{}{4}   \\[-2ex]
%                         & C     & \pointskill{}{2}  & Power BI & \pointskill{}{2}      \\[-2ex]
%                         & SQL & \pointskill{}{3}  & Wolfram Alpha    &   \pointskill{}{2}
%\end{tabular}

%\begin{tabular}{p{11em} p{1em} p{43em}}
%\skills{Aptitudes} & & Modelado matemático, capacidad de análisis, visualización de datos \\
%\skills{Comunicación} & & Español (nativo), Inglés (profesional competente)
%\end{tabular}

\subsection{Análisis de datos históricos de Selecciones Nacionales de Fútbol 
\hspace{0.1 cm} \href{https://github.com/juanjogervasio/Data-Analytics-Coderhouse}{\includegraphics[scale=0.2]{github-mark.png}}
}
Se realizó una depuración y análisis de datos públicos sobre fútbol de selecciones, disponibles en \href{https://www.kaggle.com/datasets/martj42/international-football-results-from-1872-to-2017}{kaggle.com}, con el objetivo de establecer distintos criterios para evaluar el desempeño de los equipos y establecer un orden para comparar con el del \href{https://www.kaggle.com/datasets/cashncarry/fifaworldranking}{ranking FIFA}. Se utilizaron Excel y SQL para la primera importación y transformación de los datos, y Power BI para el posterior análisis y presentación de resultados.

\vspace{0.1 cm}

\subsection{Simulación de un modelo simple de tráfico utilizando el método de Monte Carlo 
\hspace{0.1 cm} \href{https://github.com/juanjogervasio/Monte-Carlo-traffic-simulation}{\includegraphics[scale=0.2]{github-mark.png}}
}
Análisis de un modelo simple del tráfico vehicular que incluye aleatoriedad en el comportamiento de los conductores. Se simularon distintos escenarios variando las probabilidades y velocidades máximas, y se obtuvieron los diagramas fundamentales correspondientes. El código fue escrito en Python.

