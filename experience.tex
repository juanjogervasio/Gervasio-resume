%====================
% DOCENCIA INGENIERIA UNLP
%====================

% Profesor 
\subsection{{Universidad Nacional de La Plata \hfill  La Plata, Argentina}}
\subtext{Profesor Adjunto \hfill Junio 2022 --- Actualidad}
\begin{zitemize}
\item Desarrollo de clases teórico-prácticas y evaluaciones para la cátedra Matemática para Ingeniería, común a todas las carreras de la Facultad de Ingeniería. A cargo de la dirección del grupo de auxiliares docentes y la enseñanza a comisiones de entre 60 y 90 alumnos.
\item Temas principales: Álgebra, Aritmética y Cálculo de nivel inicial universitario
\item Algo del material original utilizado en las clases: \href{https://youtube.com/@gervasiojj?si=rgT2OySzyoKvKuKf}{\includegraphics[scale=0.025]{youtube.jpg}}, \href{https://drive.google.com/drive/folders/1JRLtP1kGEmH1EG_l8BVq4oRaHBvnXLOU?usp=sharing}{\includegraphics[scale=0.01]{folder.png}}
\end{zitemize}

% Jefe de Trabajos Prácticos - Exactas e Ingeniería
\subtext{Jefe de Trabajos Prácticos \hfill Agosto 2021 --- Actualidad}
\begin{zitemize}
\item Desarrollo de clases prácticas para diversas cátedras de las facultades de Ingeniería y de Ciencias.
\item Temas principales: Análisis Matemático, Física inicial universitaria, Ecuaciones diferenciales.
\item Algo del material original utilizado en las clases: \href{https://www.geogebra.org/u/juanjogervasio}{\includegraphics[scale=0.15]{Geogebra-logo.png}} , \href{https://drive.google.com/drive/folders/1O8sMBLMmO2oliVsF-9rBn2s2qDtSGJUr?usp=sharing}{\includegraphics[scale=0.01]{folder.png}}
\end{zitemize}

% Ayudante de cátedra - Exactas e Ingeniería
\subtext{Ayudante de cátedra \hfill Febrero 2015 --- Actualidad}
\begin{zitemize}
\item Auxiliar responsable de atender consultas durante las clases prácticas, en varias cátedras de las facultades de Ciencias Exactas, Informática e Ingeniería.
\item Temas principales: Álgebra básica, Análisis Matemático, Álgebra Lineal, Física inicial universitaria, Ecuaciones diferenciales.
\end{zitemize}

\vspace{0.1 cm}
%====================
% DOCENCIA EXACTAS UNLP
%====================

% Jefe de Trabajos Prácticos
%\subsection{{Universidad Nacional de La Plata / Facultad de Ciencias Exactas \hfill  La Plata, Argentina}}
%\subtext{Jefe de Trabajos Prácticos \hfill Abril 2022 --- Marzo 2024}
%\begin{zitemize}
%\item Desarrollo de clases prácticas para varias cátedras.
%\item Temas principales: Física inicial universitaria, Ecuaciones diferenciales.
%\end{zitemize}

% Ayudante de cátedra
%\subtext{Ayudante de cátedra \hfill Febrero 2015 --- Actualidad}
%\begin{zitemize}
%\item Auxiliar responsable de atender consultas durante las clases prácticas, en varias cátedras. 
%\item Temas principales: Física inicial universitaria, Análisis matemático.
%\end{zitemize}

%\vspace{0.2 cm}
%====================
% INVESTIGACION - DOCTORADO Y TESINA
%====================
\subsection{{Instituto de Física La Plata (IFLP) \hfill La Plata, Argentina}}
% Tesina - Doctorado 
\subtext{Colaborador de proyecto - Becario doctoral \hfill Marzo 2017 --- Diciembre 2020}
\begin{zitemize}
\item Analizando las aplicaciones de operadores diferenciales en problemas de Teoría Cuántica de Campos, inicialmente como parte del trabajo final de la licenciatura, y continuando como becario en el Grupo de Física Matemática del IFLP.
\item Se presentaron los resultados en forma de póster, en las 102° y 106° Reuniones anuales de la Asociación de Física Argentina, en los años 2017 y 2021: \href{https://drive.google.com/drive/folders/1NhlFmvg1QMwYczq2GqnXWXXOQsxqrctO?usp=sharing}{\includegraphics[scale=0.01]{folder.png}}
\item Temas principales: Operadores diferenciales, Funciones espectrales, Acción efectiva, Campos de Dirac.
\end{zitemize}

% Tesina
%\subtext{Colaborador de proyecto \hfill Marzo 2016 --- Marzo 2017}
%\begin{zitemize}
%\item Desarrollo de un trabajo de grado sobre la aplicación de funciones espectrales para calcular la Acción efectiva de un campo escalar sobre espacios euclídeos máximamente simétricos. 
%\item Los resultados fueron publicados como parte de una tesis de grado y como póster en una reunión científica de la especialidad. 
%\end{zitemize}
