%====================
% DOCENCIA INGENIERIA UNLP
%====================

% Profesor 
\subsection{{National University of La Plata (UNLP) \hfill  La Plata, Argentina}}
\subtext{Associate Professor \hfill June 2022 --- Present}
\begin{zitemize}
\item Developed theoretical-practical lessons and evaluations for the course Mathematics for Engineering, common to all majors in the Faculty of Engineering. In charge of directing the group of teaching assistants and teaching to groups of between 60 and 90 undergraduate students.
\item Main topics: Algebra, Arithmetic, entry-level Calculus
\item Analysis of the data from the last courses: \href{https://github.com/juanjogervasio/Intro-courses-analysis}{\includegraphics[scale=0.2]{github-mark.png}}
\item Some original learning resources: \href{https://youtube.com/@gervasiojj?si=rgT2OySzyoKvKuKf}{\includegraphics[scale=0.025]{youtube.jpg}}, \href{https://drive.google.com/drive/folders/1JRLtP1kGEmH1EG_l8BVq4oRaHBvnXLOU?usp=sharing}{\includegraphics[scale=0.01]{folder.png}}
\end{zitemize}

% Jefe de Trabajos Prácticos - Exactas e Ingeniería
\subtext{Head of Practical Assignments \hfill August 2021 --- Present}
\begin{zitemize}
\item Developed practical-oriented lessons for various chairs of the Faculties of Engineering and Exact Sciences.
\item Main topics: Calculus, entry-level Physics, Differential Equations.
\item Some original learning resources: \href{https://www.geogebra.org/u/juanjogervasio}{\includegraphics[scale=0.15]{Geogebra-logo.png}} , \href{https://drive.google.com/drive/folders/1O8sMBLMmO2oliVsF-9rBn2s2qDtSGJUr?usp=sharing}{\includegraphics[scale=0.01]{folder.png}}
\end{zitemize}

% Ayudante de cátedra - Exactas e Ingeniería
\subtext{Teaching Assistant \hfill February 2015 --- Present}
\begin{zitemize}
\item Responsible for assisting students during practical lessons, for various chairs of the Faculties of Engineering, Exact Sciences and Computer Sciences.
\item Main topics: Algebra, Calculus, Linear Algebra, entry-level Physics, Differential Equations.
\end{zitemize}

\vspace{0.1 cm}
%====================
% DOCENCIA EXACTAS UNLP
%====================

% Jefe de Trabajos Prácticos
%\subsection{{Universidad Nacional de La Plata / Facultad de Ciencias Exactas \hfill  La Plata, Argentina}}
%\subtext{Jefe de Trabajos Prácticos \hfill Abril 2022 --- Marzo 2024}
%\begin{zitemize}
%\item Desarrollo de clases prácticas para varias cátedras.
%\item Temas principales: Física inicial universitaria, Ecuaciones diferenciales.
%\end{zitemize}

% Ayudante de cátedra
%\subtext{Ayudante de cátedra \hfill Febrero 2015 --- Actualidad}
%\begin{zitemize}
%\item Auxiliar responsable de atender consultas durante las clases prácticas, en varias cátedras. 
%\item Temas principales: Física inicial universitaria, Análisis matemático.
%\end{zitemize}

%\vspace{0.2 cm}
%====================
% INVESTIGACION - DOCTORADO Y TESINA
%====================
\subsection{{Institute of Physics La Plata (IFLP) \hfill La Plata, Argentina}}
% Tesina - Doctorado 
\subtext{Research Project Collaborator \hfill March 2017 --- December 2020}
\begin{zitemize}
\item Working in the Mathematical Physics Group, analyzing the applications of differential operators to Quantum Field Theory, initially as part of my undergraduate final project, and continuing as part of my PhD studies.
\item This work was presented as a poster at the 102nd and 106th Annual Meetings of the Argentine Physics Association, in 2017 and 2021: \href{https://drive.google.com/drive/folders/1NhlFmvg1QMwYczq2GqnXWXXOQsxqrctO?usp=sharing}{\includegraphics[scale=0.01]{folder.png}}
\item Main topics: Differential operators, Spectral functions, Effective Action, Dirac and scalar fields.
\end{zitemize}

% Tesina
%\subtext{Colaborador de proyecto \hfill Marzo 2016 --- Marzo 2017}
%\begin{zitemize}
%\item Desarrollo de un trabajo de grado sobre la aplicación de funciones espectrales para calcular la Acción efectiva de un campo escalar sobre espacios euclídeos máximamente simétricos. 
%\item Los resultados fueron publicados como parte de una tesis de grado y como póster en una reunión científica de la especialidad. 
%\end{zitemize}
